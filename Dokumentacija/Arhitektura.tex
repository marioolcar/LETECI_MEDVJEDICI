\chapter{Arhitektura i dizajn sustava}

\section{Baza podataka}

\paragraph{}
{Za potrebe razvoja aplikacije SpotPicker koristi se relacijska baza podataka. 
Osnovne zadaće baze podataka su pohrana i organizacija podataka te brzo pretraživanje 
i dohvaćanje podataka kako bi ih se moglo dalje obraditi.
Svaki je entitet korištene relacijske baze naveden i opisan u daljnjem tekstu, 
a ispod opisa nalazi se tablični prikaz opisanog entiteta i njegovih atributa.}

\paragraph{}{Baza podataka aplikacije SpotPicker sastoji se od sljedećih entiteta:}
\begin{packed_item}
	\item ConformationLink
	\item Korisnik
	\item Parking
	\item Rezervacija
	\item Wallet
\end{packed_item}


\subsection{Opis tablica}

\paragraph{}
{\emph{ConformationLink\\}
Entitet ConformationLink sadrži informacije vezane za potvrđene korisnike i sadrži sljedeće atribute:
ConformationLinkID, KorisnikID, Link i isValid. Primarni ključ entiteta ConformationLink je atribut ConformationLinkID, a strani ključ je atribut KorisnikID.
Navedeni je entitet u vezi \emph{One-to-One} s entitetom Korisnik preko atributa KorisnikID.}

\paragraph{}
{\emph{Korisnik\\}
Entitet Korisnik sadrži informacije vezane za registrirane korisnike aplikacije i sadrži sljedeće atribute:
KorisnikID, Username, Password, RazinaPristupa, Name, Surname, PictureData, BankAccountNumber, Email, AccountEnabled i EmailVerified. 
Primarni ključ entiteta Korisnik je KorisnikID. 
Navedeni je entitet u vezi \emph{One-to-Many} s entitetom ConformationLink preko atributa KorisnikID, 
vezi \emph{One-to-Many} s entitetom Parking preko atributa KorisnikID, 
u vezi \emph{One-to-Many} s entitetom Rezervacija preko atributa KorisnikID 
i u vezi \emph{One-to-One} s entitetom Wallet preko atributa KorisnikID.}

\paragraph{}
{\emph{Parking\\}
Entitet Parking sadrži informacije vezane za opis i konfiguraciju parkirališta te sadrži sljedeće atribute:
ParkingID koji je primarni ključ entiteta, Name, Description, Photo, PricePerHour, Capacity i KorisnikID. Strani ključ entiteta \emph{Parking} je KorisnikID.
Navedeni je entitet u vezi \emph{Many-to-One} s entitetom Korisnik preko atributa KorisnikID, 
i u vezi \emph{One-to-Many} s entitetom Rezervacija preko atributa ParkingID.
}
	
\paragraph{}
{\emph{Rezervacija\\}
Entitet Rezervacija sadrži informacije vezanie za rezervacije pojedinih parkirnih mjesta i sadrži sljedeće atribute:
ReservationID koji je primarni ključ entiteta, KorisnikID, ParkingID, DateTimeStart, DateTimeEnd i ParkingPlaceId. Strani ključevi entiteta \emph{Rezervacija} su KorisnikID i ParkingID.
Navedeni je entitet u vezi \emph{Many-to-One} s entitetom Korisnik preko atributa KorisnikID.
}
	
\paragraph{}
{\emph{Wallet}\\
Entitet Wallet sadrži informacije vezane za novčanik korisnika i njegova sredstva. Sadrži sljedeće atribute:
WalletID koji je primarni ključ entiteta, KorisnikID koji je strani ključ entiteta i Balance.
Navedeni je entitet u vezi \emph{One-to-One} s entitetom Korisnik preko atributa KorisnikID.
}
	


\subsection{Dijagram baze podataka}
\paragraph{}
{U ovom potpoglavlju potrebno je umetnuti dijagram baze podataka. Primarni i strani kljucevi moraju biti oznaceni, a tablice povezane. Bazu podataka je potrebno normalizirati. Podsjetite se kolegija ”Baze podataka”.}



\section{Dijagram razreda i opis razreda}


\eject