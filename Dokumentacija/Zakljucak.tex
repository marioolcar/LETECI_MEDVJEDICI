\chapter{Zaključak i budući rad}
{Projekt SpotPicker predstavlja sveobuhvatnu web aplikaciju koja pruža korisnicima mogućnost jednostavne rezervacije, naplate parkiranja te pregleda slobodnih parkirališnih mjesta za automobile i bicikle. Tijekom razvoja ovog sustava, uspješno smo ostvarili zadane ciljeve, pridržavajući se specifičnih funkcionalnosti i zahtjeva postavljenih u okviru projektnog zadatka.
	
U prvoj fazi projekta, usmjerili smo se na temeljito dokumentiranje zahtjeva, koristeći se UML dijagramima i modeliranjem. Ovi dokumenti, poput obrazaca uporabe, sekvencijskih dijagrama te dijagrama razreda, bili su od ključne pomoći u daljnjem razvoju sustava. Kvalitetna priprema u prvoj fazi značajno je olakšala implementaciju u drugoj fazi, pružajući jasan smjer svakom  članu tima.
	
U drugoj fazi, članovi tima su samostalno radili na implementaciji, suočavajući se s izazovima te koristeći priliku za učenje novih alata i jezika. Kroz intenzivan rad, ostvarili smo funkcionalnosti poput rezervacije parkirališta, pregleda statistika, te integracije s OSRM API-jem za dohvat ruta. Osim toga, dovršili smo i sve potrebne UML dijagrame te izradili popratnu dokumentaciju.
	
Jedan od ključnih aspekata našeg projekta je briga o korisničkom iskustvu. Kroz jednostavno sučelje, korisnici imaju pristup svim potrebnim informacijama, mogu rezervirati parkirališta na različite načine, a voditelji parkirališta imaju uvid u statistike zauzetosti.
	
U budućnosti, mogli bismo proširiti funkcionalnosti aplikacije dodavanjem mobilne aplikacije, omogućujući korisnicima veću fleksibilnost i dostupnost. Također, kontinuirano ažuriranje sustava i praćenje povratnih informacija korisnika bit će ključno za održavanje i unaprjeđenje kvalitete aplikacije.
	
Sudjelovanje u ovom projektu pružilo nam je dragocjeno iskustvo suradnje u timu, rješavanja izazova te primjene stečenih znanja u praksi. Unatoč mogućnostima za usavršavanje, ponosni smo na postignuće i funkcionalnost SpotPicker aplikacije koja odražava naš trud i angažman u realizaciji ovog projekta.}